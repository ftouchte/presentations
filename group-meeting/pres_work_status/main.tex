\documentclass[aspectratio=169]{beamer}

\usepackage{tikz}
\usepackage{xcolor}
\usepackage{caption}
\usepackage{subcaption}
\usepackage{tabularx}
\usepackage{bm}
\usepackage{amsfonts}

\usetheme{MyStyle}   % loads beamerthemeMyStyle.sty


%%%%%%%%%%%%%%%%%%%%%%%%%%%%%%%%%%%%%%%%%%
%%%%%%%%%%%%%%%%%%%%%%%%%%%%%%%%%%%%%%%%%%
%%%%%%%%%%%%%%%%%%%%%%%%%%%%%%%%%%%%%%%%%%

\title{Experimental study of the strong interaction with the spectrometer CLAS and ALERT at JLab}
%\title{My Presentation Title}
\author{Felix Touchte Codjo}
\institute{IJCLab}
\date{\today}


\begin{document}

\begin{frame}[plain] % remove header, footer, barre de navigation
\titlepage
\end{frame}

%\begin{frame}{Outline}
%\tableofcontents
%\end{frame}

\section{ALERT experiment}

\begin{frame}{ALERT experiment}
    \begin{itemize}
        \item ALERT is a detector. It stands for \alert{A} \alert{L}ow \alert{E}nergy \alert{R}ecoil \alert{T}agger
            \begin{itemize}
                \item a hyperbolic drift chamber (AHDC) $\rightarrow$ IJClab, France
                \item + a time-of-flight (ATOF) $\rightarrow$ Argonne Laboratory, US
            \end{itemize}
        \item It is also an experiment with a wide program in nuclear physics
            \begin{itemize}
                \item e.g. \alert{Deeply Virtual Compton Scattering (DVCS) on $^4\mbox{He}$}
            \end{itemize}
        \item The experiment took place at JLab from April 2025 to September 2025
        %\item 
        \begin{figure}
            \centering
            \begin{subfigure}{0.35\textwidth}
                \includegraphics[width=\linewidth]{figures/alert-detector.png}
            \end{subfigure}
            \hspace{0.05\textwidth}
            \begin{subfigure}{0.35\textwidth}
                \includegraphics[width=\linewidth]{figures/build/feynam-diagram-dvcs-he4.pdf}
            \end{subfigure}
        \end{figure}
    \end{itemize}
\end{frame}

\begin{frame}{Run period at JLab}
    \begin{itemize}
        \item The experimental setup was the following:
        \begin{itemize}
            \item The electron beam is delivered by CEBAF
            \item The target is located at the center of ALERT
            \item The low recoil fragments are detected by ALERT
            \item The scattered electrons and produced photons are detected by CLAS12
        \end{itemize}
    \end{itemize}
    \begin{figure}
        \centering
        \begin{subfigure}{0.28\textwidth}
            \includegraphics[width=\linewidth]{figures/cebaf.png}
            \caption{CEBAF, can deliver spin polarized electron beam with energy up to 11 GeV}
        \end{subfigure}
        \hspace{0.01\textwidth}
        \begin{subfigure}{0.28\textwidth}
            \includegraphics[width=\linewidth]{figures/build/feynam-diagram-dvcs-he4.pdf}
            \caption{example of physics process : nuclear DVCS on $^4\mbox{He}$}
        \end{subfigure}
        \hspace{0.01\textwidth}
        \begin{subfigure}{0.28\textwidth}
            \includegraphics[width=\linewidth]{figures/clas12.png}
            \caption{CLAS12, good $e^-$ and $\gamma$ identification over a large kinematics coverage}
        \end{subfigure}
    \end{figure}
\end{frame}

\begin{frame}{Run period at JLab}
    \begin{itemize}
        \item We took data on several nuclear targets (H2, D2, He4) and with different beam energies (2 GeV, 6 GeV, 10.6 GeV) 
        \item I was at JLab during a good part the run period (until July 2025)
            \begin{itemize}
                \item Target expert (target change every one or two weeks, target purge every day)
                \item ALERT expert (monitor the quality of the data taking, on-call person)
            \end{itemize}
        \item The data taking was a success, we achieved our goal in terms of accumulated charge
        \item \alert{Now, all efforts are dedicated the finalization of the reconstruction software}
    \end{itemize}
    
\end{frame}

\section{Simulation}

\begin{frame}{Simulation}
    \begin{itemize}
        \item Generate the characteristic signal of the AHDC from the Geant4 digitization
            \begin{itemize}
                \item collection of hits $\{(E_s, x_s, y_s, z_s, p_{x_s}, ...)\}_s \longrightarrow$ signal over the time 
            \end{itemize}
    \end{itemize}
    \begin{figure}
        \centering
        \begin{subfigure}{0.35\textwidth}
            \includegraphics[width=\linewidth]{figures/build/ahdc_cell.pdf}
            \caption{AHDC detection cell}
        \end{subfigure}
        \hspace{0.05\textwidth}
        \begin{subfigure}{0.45\textwidth}
            \includegraphics[width=\linewidth]{figures/build/ahdc_signal_landau.pdf}
            \caption{Shape of the AHDC signal}
        \end{subfigure}
    \end{figure}
\end{frame}

\begin{frame}{Simulation}
    \begin{itemize}
        \item We have identified few control parameters that have to be calibrated in order to match real data
        \begin{itemize}
            \item (inverse of) time2distance
            \item reference time ($t_0$)
            \item signal width parameter (for a Landau distribution, it is linked to the timeOverThreshold)
            \item proportionality factor (conversion of MeV to ADC)
            \item noise level
        \end{itemize}
    \end{itemize}
    \begin{figure}
        \centering
        \begin{subfigure}{0.25\textwidth}
            \includegraphics[width=\linewidth]{figures/build/ahdc_cell.pdf}
        \end{subfigure}
        \hspace{0.05\textwidth}
        \begin{subfigure}{0.35\textwidth}
            \includegraphics[width=\linewidth]{figures/build/ahdc_signal_landau.pdf}
        \end{subfigure}
    \end{figure}
\end{frame}

\begin{frame}{Simulation (results)}
    \begin{itemize}
        \item The distributions look very similar (the peaks are located at the same place)
        \item Simulation (top) versus real data (bottom) for elastics deuterons (run 22712)
        \item We have overestimated the noise in simulation, but we didn't manage to spread the distribution as in real data
        \item Taking into account the calibration of the amplitude and of the timeOverThreshold in the future could improve the result for real data
    \end{itemize}
    \begin{figure}
        \centering
        \begin{subfigure}{0.242\textwidth}
            \includegraphics[width=\linewidth]{figures/simu-deuteron-signal.pdf}
        \end{subfigure}
        \begin{subfigure}{0.242\textwidth}
            \includegraphics[width=\linewidth]{figures/simu-deuteron-amplitude.pdf}
        \end{subfigure}
        \begin{subfigure}{0.242\textwidth}
            \includegraphics[width=\linewidth]{figures/simu-deuteron-timeOverThreshold.pdf}
        \end{subfigure}
        \begin{subfigure}{0.242\textwidth}
            \includegraphics[width=\linewidth]{figures/simu-deuteron-time.pdf}
        \end{subfigure}

        \vspace{0.2cm}

        \begin{subfigure}{0.242\textwidth}
            \includegraphics[width=\linewidth]{figures/data-deuteron-signal.pdf}
        \end{subfigure}
        \begin{subfigure}{0.242\textwidth}
            \includegraphics[width=\linewidth]{figures/data-deuteron-amplitude.pdf}
        \end{subfigure}
        \begin{subfigure}{0.242\textwidth}
            \includegraphics[width=\linewidth]{figures/data-deuteron-timeOverThreshold.pdf}
        \end{subfigure}
        \begin{subfigure}{0.242\textwidth}
            \includegraphics[width=\linewidth]{figures/data-deuteron-time.pdf}
        \end{subfigure}
    \end{figure}
\end{frame}

\section{Decoding}

\begin{frame}{Decoding}
    \begin{itemize}
        \item The decoding consists of the extraction of the \textbf{leadingEdgeTime}, \textbf{timeOverThreshold}, \textbf{pedestal} and \textbf{amplitude} from a given AHDC pulse
        \item Even in simulation, the quality of the decoding varies with the amplitude of the signal (deltaTime = true - reconstructed)
        \item Proton signals have very low amplitudes (~ 200 ADC), how can we trust the decoding?
    \end{itemize}
    \begin{figure}
        \centering
        \begin{subfigure}{0.45\textwidth}
            \includegraphics[width=\linewidth]{figures/simu-deuteron-signal.pdf}
        \end{subfigure}
        \hspace{0.02\textwidth}
        \begin{subfigure}{0.45\textwidth}
            \includegraphics[width=\linewidth]{figures/simu-deuteron-deltaTime_amplitude.pdf}
        \end{subfigure}
    \end{figure}
\end{frame}

\begin{frame}{Waveform classification}
    \begin{itemize}
        \item We have identified seven different patterns for the AHDC signals
    \end{itemize}
    \hspace{0.5cm}
    \begin{minipage}{0.45\textwidth}
        \begin{itemize}
            \small
            \item[$\circ$] 6 $\rightarrow$ too short (nsamples $\leq$ 10)
            \item[$\circ$] 5 $\rightarrow$ decreasing baseline
            \item[$\circ$] 4 $\rightarrow$ bad ToT (ToT $<$ 300)
            \item[$\circ$] 3 $\rightarrow$ pileup (not done yet)
        \end{itemize}
    \end{minipage}
    \begin{minipage}{0.4\textwidth}
        \begin{itemize}
            \small
            \item[$\circ$] 2 $\rightarrow$ bad trailingEdgeTime
            \item[$\circ$] 1 $\rightarrow$ saturating
            \item[$\circ$] 0 $\rightarrow$ good
            \item[] 
        \end{itemize}
        
    \end{minipage}

    \begin{figure}
        \raggedright
        \begin{subfigure}{0.242\textwidth}
            \includegraphics[width=\linewidth]{figures/wfType6.pdf}
        \end{subfigure}
        \begin{subfigure}{0.242\textwidth}
            \includegraphics[width=\linewidth]{figures/wfType5.pdf}
        \end{subfigure}
        \begin{subfigure}{0.242\textwidth}
            \includegraphics[width=\linewidth]{figures/wfType4.pdf}
        \end{subfigure}
        \begin{subfigure}{0.242\textwidth}
            \includegraphics[width=\linewidth]{figures/wfType3.pdf}
        \end{subfigure}

        \vspace{0.2cm}
        
        \begin{subfigure}{0.242\textwidth}
            \includegraphics[width=\linewidth]{figures/wfType2.pdf}
        \end{subfigure}
        \begin{subfigure}{0.242\textwidth}
            \includegraphics[width=\linewidth]{figures/wfType1.pdf}
        \end{subfigure}
        \begin{subfigure}{0.242\textwidth}
            \includegraphics[width=\linewidth]{figures/wfType0.pdf}
        \end{subfigure}
    \end{figure}
\end{frame}

\begin{frame}{Hit selection}
    \begin{itemize}
        \item Criteria: wfType $\leq$ 2 and quality cuts
        \item Run 23003, beam current 200 nA
    \end{itemize}
    \begin{figure}
        \centering
        \includegraphics[width=0.72\linewidth]{figures/hit-selection-evolution.png}
    \end{figure}
\end{frame}

% \begin{frame}{Calibration (time2distance)}
%     \begin{itemize}
%         \item Haha
%     \end{itemize}
%     \begin{figure}
%         \centering
%         \begin{subfigure}{0.32\textwidth}
%             \includegraphics[width=\linewidth]{figures/residual_pre_time2distance_calib.pdf}
%         \end{subfigure}
%         \begin{subfigure}{0.32\textwidth}
%             \includegraphics[width=\linewidth]{figures/time2distance.pdf}
%         \end{subfigure}
%         \begin{subfigure}{0.32\textwidth}
%             \includegraphics[width=\linewidth]{figures/residual_post_time2distance_calib.pdf}
%         \end{subfigure}
%     \end{figure}
    
% \end{frame}

\section{Kalman Filter}

\begin{frame}{Kalman Filter (principle)}
    \begin{itemize}
        \item Give an estimation of the true state $\bm{x}_k$ knowing a series of measurements $\bm{z}_1...\bm{z}_k$
            \begin{equation*}
                \bm{\hat{x}}_{k|k} = \mathbb{E}[\bm{x}_k | \bm{z}_1...\bm{z}_k]
            \end{equation*}
        \item Conditions:
            \begin{itemize}
                \item a discrete evolution model
                    \begin{equation*}
                        \bm{x_k} = f(\bm{x}_{k-1}, u_{k-1}, w_{k-1}) 
                    \end{equation*}
                \item an expression of the measurement as function of the state
                    \begin{equation*}
                        \bm{z_k} = h(\bm{x}_k, v_{k-1}) 
                    \end{equation*}
            \end{itemize}
        \item $w_k$, $v_k$ are respectively the process noise and the measurement noise; $u_k$ is an optional control input
    \end{itemize}
    
\end{frame}

\begin{frame}{Kalman Filter (algorithm)}
    \begin{itemize}
        \item The Kalman Filter is recursive
        \item During each update, the Kalman filter minimize the error covariance matrix
            \begin{equation*}
                \bm{P}_{k|k} = \mathbb{E}[(\bm{x}_k - \bm{\hat{x}}_{k|k})(\bm{x}_k - \bm{\hat{x}}_{k|k})^T]
            \end{equation*}
        \item \alert{Figure to be updated to match the notations}
    \end{itemize}
    \begin{figure}
        \centering
        \includegraphics[width=0.7\linewidth]{figures/tmp-kalman-filter-algo.png}
    \end{figure}
\end{frame}

\begin{frame}{Kalman Filter (visualisation)}
    \begin{itemize}
        \item This simulation. The system evolves from left to right. Niter = 60 (this example).
    \end{itemize}
    \begin{figure}
        \centering
        \includegraphics[width=0.9\linewidth]{figures/kalman-filter-monitoring-simu.pdf}
    \end{figure}
\end{frame}

\begin{frame}{Kalman Filter (AHDC case)}
    \begin{itemize}
        \item For the AHDC, the state vector is $\bm{x} = (x, y, z, p_x, p_y, p_z)^T$
        \item The measurements are
            \begin{itemize}
                \item the distances provided by the hits belonging to a given track, here $z = d$ is $1\times1$ matrix
                \item the position with respect to the beamline, here $\bm{z} = (r, \phi, z)^T$ is a $3\times1$ matrix
            \end{itemize}
        \item The evolution model is the one of a particle moving in an electromagnetic field
        \vspace{0.5cm}
        \item The first implementation of the algorithm has been done by Mathieu Ouillon and Éric Fuchey
        \item My contributions: cleaning, reorganization, new features
    \end{itemize}
    
\end{frame}

\begin{frame}{Kalman Filter (new features)}
    \begin{enumerate}
        \item \textbf{Use all hits} (before the code didn't manage more than one hit on the same layer)
            \begin{itemize}
                \item[$\circ$] the effect is more visible in simulation
                \item[$\circ$] a good part of the peak at 0 in figure (a) is due all the hits that are not used
                \item[$\circ$] looking at (b), almost 6.8 \% of the information was missing
            \end{itemize}
    \end{enumerate}
    \begin{figure}
        \centering
        \begin{subfigure}{0.45\textwidth}
            \includegraphics[width=\linewidth]{figures/residual_before_use_all_hits_residual.pdf}
            \caption{before}
        \end{subfigure}
        \begin{subfigure}{0.45\textwidth}
            \includegraphics[width=\linewidth]{figures/residual_after_use_all_hits_residual.pdf}
            \caption{after}
        \end{subfigure}
    \end{figure}
    
\end{frame}

\begin{frame}{Kalman Filter (new features)}
    \begin{enumerate}
        \item[2.] \textbf{Implement the electron vertex}
            \begin{itemize}
                \item[$\circ$] read electron vertex from CLAS12 reconstruction
                \item[$\circ$] set it to the track with a finite resolution
                \item[$\circ$] this resolution will depend on the momentum $p_e$ and the $\theta_e$ angle of the electron (\alert{a fine-tuning will be done later}) 
                \item[$\circ$] handle a possible misalignment between the center of CLAS and ALERT
            \end{itemize}
        \item[3.] \textbf{Make the distance resolution dependent on the time and on the amplitude}
            \begin{itemize}
                \item[$\circ$] \alert{a fine-tuning will be done later}
            \end{itemize}
        \item[4.] \alert{(Ongoing work)} \textbf{Make the Kalman Filter reusable}
            \begin{itemize}
                \item[$\circ$] e.g. \textbf{KF 1} $\rightarrow$ clean bad hits $\rightarrow$ \textbf{KF 2}
            \end{itemize}
        \item[5.] \alert{(Next step)} Add the ATOF hit in the Kalman Filter
    \end{enumerate}
    
\end{frame}

\begin{frame}{Reconstruction status}
    \begin{itemize}
        \item Run 22712 on D2 target, 2.23951 GeV beam energy
        \item Elastic cuts: $3.5 \mbox{ GeV}^2< W^2 < 3.8 \mbox{ GeV}^2$ and $\Delta \phi < 20^\circ$
        \item Deposited energy dEdx versus momentum p of the track (after the Kalman filter)
    \end{itemize}
    \begin{figure}
        \centering
        \includegraphics[width=0.75\linewidth]{figures/recon-status-v85-dEdx-versus-momentum.pdf}
    \end{figure}
\end{frame}

\begin{frame}{Conclusion}
    \begin{itemize}
        \item A lot of software development
        \item ...
        \item (After the presentation), points non abordés : time2distance, geometry modification, Nombre d'itérations et computing time du Kalman réduit en améliorant le Propagator...
    \end{itemize}
    
\end{frame}

\section{Backup}
\begin{frame}{Backup slides}
    \begin{figure}
        \centering
        \begin{subfigure}{0.48\textwidth}
            \includegraphics[width=\linewidth]{figures/recon-status-v85-missing-mass.pdf}
            \caption{Missing mass}
        \end{subfigure}
        \begin{subfigure}{0.48\textwidth}
            \includegraphics[width=\linewidth]{figures/recon-status-v85-delta-phi.pdf}
            \caption{Delta Phi}
        \end{subfigure}
    \end{figure}
\end{frame}

\begin{frame}{Calibration (time2distance)}
    \begin{itemize}
        \item Haha
    \end{itemize}
    \begin{figure}
        \centering
        \begin{subfigure}{0.32\textwidth}
            \includegraphics[width=\linewidth]{figures/residual_pre_time2distance_calib.pdf}
        \end{subfigure}
        \begin{subfigure}{0.32\textwidth}
            \includegraphics[width=\linewidth]{figures/time2distance.pdf}
        \end{subfigure}
        \begin{subfigure}{0.32\textwidth}
            \includegraphics[width=\linewidth]{figures/residual_post_time2distance_calib.pdf}
        \end{subfigure}
    \end{figure}
    
\end{frame}

% \begin{frame}{Updates}
%     \begin{itemize}
%     \item For now, moving the Kalman Filter in ALERTEngine has no effect on the result but:
%         \begin{itemize}
%             \item the new implementation allows us to use it several times
%             \item example:  \textbf{KF 1 $\rightarrow$ clean bad hits $\rightarrow$ KF 2}
%         \end{itemize}
%     \item Also, it was a mistake at the beginning, but now these quantities can be filled independently (the values for the Helix fitter are empty.)
%     \end{itemize}
%     \begin{figure}
%         \centering
%         \begin{subfigure}{0.3\textwidth}
%             \begin{tikzpicture}
%                 % Image de fond
%                 \node[anchor=south west] (img) at (0,0) {\includegraphics[width=\linewidth]{figures/trackBank.png}};
%                 % Coordonnées normalisées 0–1
%                 \begin{scope}[x={(img.south east)}, y={(img.north west)}]
%                     \draw[red, ultra thick] (0.05,0.05) rectangle (0.7, 0.48);
%                     %\node[blue] at (0.6,0.55) {Point important};
%                 \end{scope}
%             \end{tikzpicture}
%         \end{subfigure}
%         \hspace{0.01\textwidth}
%         \begin{subfigure}{0.3\textwidth}
%             \begin{tikzpicture}
%                 % Image de fond
%                 \node[anchor=south west] (img) at (0,0) {\includegraphics[width=\linewidth]{figures/kftrackBank.png}};
%                 % Coordonnées normalisées 0–1
%                 \begin{scope}[x={(img.south east)}, y={(img.north west)}]
%                     \draw[red, ultra thick] (0.05,0.05) rectangle (0.7, 0.43);
%                     %\node[blue] at (0.6,0.55) {Point important};
%                 \end{scope}
%             \end{tikzpicture}
%         \end{subfigure}
%     \end{figure}
% \end{frame}


% \begin{frame}{Reconstruction status}
%     \begin{itemize}
%         \item Run 22712 on D2 target, 2.23951 GeV beam energy
%         \item Elastic cuts: $3.4 \mbox{ GeV}^2< W^2 < 3.9 \mbox{ GeV}^2$ and $\Delta \phi < 45^\circ$
%         \item Deposited energy dEdx versus momentum p of the track (after the Kalman filter)
%     \end{itemize}
%     \begin{figure}
%         \centering
%         \includegraphics[width=0.6\linewidth]{figures/recon-status.pdf}
%     %\caption{}
%     \end{figure}
% \end{frame}

\end{document}


