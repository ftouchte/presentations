\documentclass[aspectratio=169]{beamer}

\usepackage{tikz}
\usepackage{xcolor}
\usepackage{caption}
\usepackage{subcaption}
\usepackage{bm}

\usetheme{MyStyle}   % loads beamerthemeMyStyle.sty


%%%%%%%%%%%%%%%%%%%%%%%%%%%%%%%%%%%%%%%%%%
%%%%%%%%%%%%%%%%%%%%%%%%%%%%%%%%%%%%%%%%%%
%%%%%%%%%%%%%%%%%%%%%%%%%%%%%%%%%%%%%%%%%%

\title{ALERT meeting - Kalman Filter \\ updates}
%\title{My Presentation Title}
\author{Felix Touchte Codjo}
\institute{IJCLab}
\date{\today}


\begin{document}

\begin{frame}[plain] % remove header, footer, barre de navigation
\titlepage
\end{frame}

%\begin{frame}{Outline}
%\tableofcontents
%\end{frame}

%\section{Intro}

\begin{frame}{Updates}
    \begin{itemize}
        \item Cleaning of bad hits
        \begin{itemize}
            \item Made the Kalman Filter reusable (e.g the error matrixes are saved from one use to another)
            \item Feature: we run KF once, then we remove all hits with $\bm{|\mbox{residual}| > 1.5 \mbox{ mm}}$, and finally we rerun the KF with 15 iterations
            \item \alert{The standard deviation is better}
        \end{itemize}
    \end{itemize}
    \begin{figure}
        \begin{subfigure}{0.45\textwidth}
            \includegraphics[width=\linewidth]{figures/residual-before-hit-cleaning.pdf}
            \caption{before}
        \end{subfigure}
        \hspace{0.01\textwidth}
        \begin{subfigure}{0.45\textwidth}
            \includegraphics[width=\linewidth]{figures/residual-after-hit-cleaning.pdf}
            \caption{after}
        \end{subfigure}
    \end{figure}
\end{frame}

\begin{frame}{Ongoing work - ATOF hits}
    \begin{itemize}
        \item Include the ATOF hit in the Kalman Filter
            \begin{itemize}
                \item Use \textbf{ALERT::ai:projections} \texttt{\{trakid, matched\_atof\_hit\_id\}}
                \item nb of matched wegdes / nb of tracks = $\bm{20.94 \%}$
                    \begin{itemize}
                        \item[$\rightarrow$] nb bars with the same layer id = $\bm{10 \%}$
                        \item[$\rightarrow$] nb bars with the same layer id or $\pm 1$ = $\bm{20.53 \%}$
                    \end{itemize}
            \end{itemize}
        \item I already have a first implementation in coatjava
            \begin{itemize}
                \item But, I still need reasonable estimations of the resolution in $z$, $r$, $\phi$
                \item Any clue?
                \item $\delta r^2 \rightarrow$  (9 mm$^2$), $\delta \phi^2 \rightarrow$  (9 deg$^2$), $\delta z^2 \rightarrow$ ?
            \end{itemize}
    \end{itemize}
\end{frame}

% \begin{frame}{Ongoing work - ATOF hits}
%     \begin{itemize}
%         \item Question on the z resolution of the ATOF bar?
%         \begin{figure}
%             \centering
%             \begin{subfigure}{0.3\textwidth}
%                 \includegraphics[width=\linewidth]{figures/mcParticle.png}
%                 \caption{MC::Particle}
%             \end{subfigure}
%             \begin{subfigure}{0.3\textwidth}
%                 \includegraphics[width=\linewidth]{figures/atofHits.png}
%                 \caption{ATOF::hits}
%             \end{subfigure}
%             \begin{subfigure}{0.3\textwidth}
%                 \includegraphics[width=\linewidth]{figures/kfTrack.png}
%                 \caption{AHDC::kftrack}
%             \end{subfigure}
%         \end{figure}
%     \end{itemize}
% \end{frame}

\begin{frame}{Reconstruction status}
    \begin{itemize}
        \item dEdx verus p, run 22712 on D2
        \item Elastic cuts: $3.5 \mbox{ GeV}^2 < W^2 < 3.8 \mbox{ GeV}^2$, $|\Delta \phi| < 20^\circ$, $\mbox{nhits} \geq 6$
        \item Hit cleaning included (work on the ATOF hit not included)
    \end{itemize}
    \begin{figure}
        \centering
        \includegraphics[width=0.8\linewidth]{figures/recon-status-dEdx-versus-momentum.pdf}
    \end{figure}
\end{frame}

% \begin{frame}{Updates}
%     \begin{itemize}
%     \item For now, moving the Kalman Filter in ALERTEngine has no effect on the result but:
%         \begin{itemize}
%             \item the new implementation allows us to use it several times
%             \item example:  \textbf{KF 1 $\rightarrow$ clean bad hits $\rightarrow$ KF 2}
%         \end{itemize}
%     \item Also, it was a mistake at the beginning, but now these quantities can be filled independently (the values for the Helix fitter are empty.)
%     \end{itemize}
%     \begin{figure}
%         \centering
%         \begin{subfigure}{0.3\textwidth}
%             \begin{tikzpicture}
%                 % Image de fond
%                 \node[anchor=south west] (img) at (0,0) {\includegraphics[width=\linewidth]{figures/trackBank.png}};
%                 % Coordonnées normalisées 0–1
%                 \begin{scope}[x={(img.south east)}, y={(img.north west)}]
%                     \draw[red, ultra thick] (0.05,0.05) rectangle (0.7, 0.48);
%                     %\node[blue] at (0.6,0.55) {Point important};
%                 \end{scope}
%             \end{tikzpicture}
%         \end{subfigure}
%         \hspace{0.01\textwidth}
%         \begin{subfigure}{0.3\textwidth}
%             \begin{tikzpicture}
%                 % Image de fond
%                 \node[anchor=south west] (img) at (0,0) {\includegraphics[width=\linewidth]{figures/kftrackBank.png}};
%                 % Coordonnées normalisées 0–1
%                 \begin{scope}[x={(img.south east)}, y={(img.north west)}]
%                     \draw[red, ultra thick] (0.05,0.05) rectangle (0.7, 0.43);
%                     %\node[blue] at (0.6,0.55) {Point important};
%                 \end{scope}
%             \end{tikzpicture}
%         \end{subfigure}
%     \end{figure}
% \end{frame}

\end{document}


