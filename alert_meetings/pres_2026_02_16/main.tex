\documentclass[aspectratio=169]{beamer}

\usepackage{tikz}
\usepackage{xcolor}
\usepackage{caption}
\usepackage{subcaption}
\usepackage{bm}

\usetheme{MyStyle}   % loads beamerthemeMyStyle.sty


%%%%%%%%%%%%%%%%%%%%%%%%%%%%%%%%%%%%%%%%%%
%%%%%%%%%%%%%%%%%%%%%%%%%%%%%%%%%%%%%%%%%%
%%%%%%%%%%%%%%%%%%%%%%%%%%%%%%%%%%%%%%%%%%

\title{ALERT meeting - Kalman Filter \\ updates}
%\title{My Presentation Title}
\author{Felix Touchte Codjo}
\institute{IJCLab}
\date{\today}


\begin{document}

\begin{frame}[plain] % remove header, footer, barre de navigation
\titlepage
\end{frame}

%\begin{frame}{Outline}
%\tableofcontents
%\end{frame}

%\section{Intro}

\begin{frame}{Updates}
    \begin{itemize}
        \item Cleaning of bad hits
        \begin{itemize}
            \item After a first use the Kalman Filter, we remove all hits with $\bm{|\mbox{residual}| > 1.5 \mbox{ mm}}$
            \item Then we rerun the Kalman Filter with 15 iterations
            \item \alert{The standard deviation is better}
        \end{itemize}
    \end{itemize}
    \begin{figure}
        \begin{subfigure}{0.45\textwidth}
            \includegraphics[width=\linewidth]{figures/residual-before-hit-cleaning.pdf}
        \end{subfigure}
        \hspace{0.01\textwidth}
        \begin{subfigure}{0.45\textwidth}
            \includegraphics[width=\linewidth]{figures/residual-after-hit-cleaning.pdf}
        \end{subfigure}
    \end{figure}
\end{frame}

% \begin{frame}{Updates}
%     \begin{itemize}
%     \item For now, moving the Kalman Filter in ALERTEngine has no effect on the result but:
%         \begin{itemize}
%             \item the new implementation allows us to use it several times
%             \item example:  \textbf{KF 1 $\rightarrow$ clean bad hits $\rightarrow$ KF 2}
%         \end{itemize}
%     \item Also, it was a mistake at the beginning, but now these quantities can be filled independently (the values for the Helix fitter are empty.)
%     \end{itemize}
%     \begin{figure}
%         \centering
%         \begin{subfigure}{0.3\textwidth}
%             \begin{tikzpicture}
%                 % Image de fond
%                 \node[anchor=south west] (img) at (0,0) {\includegraphics[width=\linewidth]{figures/trackBank.png}};
%                 % Coordonnées normalisées 0–1
%                 \begin{scope}[x={(img.south east)}, y={(img.north west)}]
%                     \draw[red, ultra thick] (0.05,0.05) rectangle (0.7, 0.48);
%                     %\node[blue] at (0.6,0.55) {Point important};
%                 \end{scope}
%             \end{tikzpicture}
%         \end{subfigure}
%         \hspace{0.01\textwidth}
%         \begin{subfigure}{0.3\textwidth}
%             \begin{tikzpicture}
%                 % Image de fond
%                 \node[anchor=south west] (img) at (0,0) {\includegraphics[width=\linewidth]{figures/kftrackBank.png}};
%                 % Coordonnées normalisées 0–1
%                 \begin{scope}[x={(img.south east)}, y={(img.north west)}]
%                     \draw[red, ultra thick] (0.05,0.05) rectangle (0.7, 0.43);
%                     %\node[blue] at (0.6,0.55) {Point important};
%                 \end{scope}
%             \end{tikzpicture}
%         \end{subfigure}
%     \end{figure}
% \end{frame}

\end{document}


